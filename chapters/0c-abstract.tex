\chapter*{Abstract}

The shipping industry is a vast and complex trading system that is capitally intensive, involves many companies and businesses, and is generally believed to be responsible for around 90\% of all world trade \parencite{maritime_studies@2011}. Interested parties are all continuously searching for accurate information that can help them understand the future ebbs and flows of this volatile market that primarily consists of cargo demand and vessel supply. Thus, being able to effectively predict future movements and the availability of shipping vessels can be essential for many of the people involved in the industry.

Although the industry has traditionally relied on non-digital services, in recent years, there has been an increase in available software solutions that aims to assist shipping businesses in their decision-making processes. Many of these software products are based on the availability of Automated Identification System (AIS) data. AIS has become a globally adopted standard enforced by the International Maritime Organization (IMO) since 2006 for safety and navigation reasons. However, since AIS transmitters emit all commercial vessels' navigational data, it also has commercial value in that it provides a global overview of shipping vessels' movements over time. Recent studies into historical AIS data further elaborates that it is indeed applicable toward predicting future trajectories and movements of vessels and that Machine Learning (ML) techniques can be applied to this topic area.

This thesis investigates the area of vessel destination prediction and proposes a Machine Learning (ML) approach based on a combination of historical AIS data and technical vessel details such as vessel type, or segments. The proposed model applies to any vessel, is unrestricted by time or geographical limitations, and achieved an accuracy level of \textit{72\%} depending on vessel segments and sub-segments. The thesis was written in collaboration with the maritime tech startup company Maritime Optima (MO) who provided the initial data foundation used to develop the proposed method.
