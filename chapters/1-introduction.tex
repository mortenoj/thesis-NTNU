\chapter{Introduction}

\section{Topics covered by project}
\label{sec:topics_covered}

The topics covered by this project mainly include applying computer science techniques to the problem of predicting shipping vessels' future destinations and \gls{voyage} patterns to assist various actors in the shipping industry in their daily decision-making processes. More specifically, the thesis focuses on the aspect of applying \acrfull{ml} techniques to vessel destination prediction using different sources of vessel information such as \acrfull{ais}, \gls{voyage} patterns, and individual vessel information such as vessel types, or segments. The goal of the thesis is to establish a high-quality, general prediction method not restricted by geographical extent or specific time intervals, and to discuss possible applications and value of the model in the current state of the art of the shipping industry.

\section{Keywords}

AIS data, vessel destination prediction, vessel supply, machine learning, maritime logistics

\section{Problem description}
\label{sec:problem_desc}

Many, or most, companies in the shipping industry heavily rely on predicting the market in order to optimize their return of investment (ROI) and generally make smarter decisions resulting in beneficial investments. The market is generally defined by supply and demand where, in this case, demand consists of available cargoes to be shipped, and supply consists of vessels available to ship the cargoes. \cref{fig:maritime_economics} shows how different factors influence investment cycles within the shipping industry and the general market. Furthermore, the current methods for gathering data and conducting analysis are normally manual and paid services provided by specialists called brokers. The industry is still prone to using non-digital methods and external services to provide relevant information about vessel supply and traveling patterns.

\begin{figure}[htbp]
    \centering
    \includegraphics[width=0.9\textwidth]{figures/investment_cycle}
    \caption{Vessel supply’s role in the shipping market and investment cycles \parencite{stopford2008}}
    \label{fig:maritime_economics}
\end{figure}

The data required to make good predictions are generally considered proprietary in the industry which is hesitant to share information. However, in recent years, vessel information has become more available through the \acrshort{ais} standard that provides information including vessel positions, navigational statuses, and manually inputted \gls{voyage} information. In 2004, the \acrfull{imo} initiated the \acrshort{ais} protocol which all commercial vessels over 299 \acrfull{gt} are required to use. This serves as a plentiful source of information applicable toward the analysis of vessel availability on a global scale.

Although the usage of \acrshort{ais} has been enforced and globally adopted, manually inputted data within the protocol lacks standardization. These attributes of the \acrshort{ais} protocol includes non-navigational \gls{voyage} related information such as the vessel's intended destination and \acrfull{eta}. In contrast, the positional and navigational information within the protocol is automized, and therefore mostly accurate. The manually inputted information is managed by the vessel's crew or captain and is therefore prone to human error in regards to either format or misinformation. \cite{mestl2016} claims the accuracy of this information to be as low as 4\% in certain areas. To use \acrshort{ais} data, existing prediction methods, therefore only consider the geographical information that is automated, mainly geographical coordinates similar to that of GPS\@, speed, and heading. On the other hand, other aspects such as vessel type, dimensions, and draft, have extensively been overlooked in such methods which limits them in terms of accuracy when applied to a general range of vessels. Therefore, this thesis proposes an approach to vessel destination predictions that takes advantage of a broader range of vessel and voyage information to construct a reliable and generally applicable prediction method.

\section{Justifications, motivation, and benefits}
\label{section:justifications_motivations_benefits}

The shipping industry is a vast industry that affects the entire world. It is generally believed to be responsible for \textit{90\%} of all world trade \parencite{maritime_studies@2011} but is also a massive contributor to global air pollution which negatively affects the environment \parencite{zheng2016:online}. However, because of the ever-increasing global demand for products and services, it is presumable that the importance of the shipping industry will only increase in the future. This excludes reduction of shipping activities as a viable option, but it leaves room for innovation in terms of optimization since even small improvements on voyage routes and traveling patterns can have huge implications for both revenue and environmental impact. Furthermore, because of the vast volume of vessels and their cargo capacities, shipping investments generate a massive amount of revenue. For individual investors and companies, being able to rely on market predictions is key to making beneficial investments. As an example, on 23 March 2021, one of the largest container vessels in the world, Ever Given, ran aground in the Suez Canal. This event was publicized worldwide because of the blockage's immense impact. Some estimates say the blockage cost on global trade lied between 6 and 10 billion USD\footnote{\url{https://www.bbc.com/news/business-56559073}}, signaling the tangible impact of the shipping industry on the global economy as a whole.

Although there has been considerable research into vessel destination and trajectory predictions, the current literature appears to focus on smaller-scale predictions that emphasise topics such as collision avoidance and anomaly detection (\cref{sec:lit_review}). Furthermore, as mentioned in \cref{sec:problem_desc}, existing works extensively overlook specific vessel details in favor of analyzing the geographical information provided by \acrshort{ais}. Of the research that offers more general predictions, such as forecasting the availability of vessels, efforts in this direction have been comparatively limited. The paper~\cite{lechtenberg2019} which was presented at the Hamburg International Conference of Logistics (HICL) in 2019 claims: \textit{“Regarding the forecast of ship-supply so far --- to the best of our knowledge --- no research has investigated possibilities to predict the number of available ships in a certain region of interest.”} which supports the observation made above.

To enable research in this direction, as part of this thesis, the collaborating company \acrfull{mo} provides high-quality historical \acrshort{ais} data in a highly available format and has already employed systems that can detect vessel arrivals and departures from a global set of shipping ports. This enables the thesis to focus more on analysis and applications rather than data collection and validation. Lastly, the thesis author has been employed at \acrshort{mo} since the founding of the company and has been contributing to the development of their digital platform ever since. These factors combined are the main motivating factors behind this thesis.

\section{Research questions}
\label{sec:research_questions}

The main research question the thesis aims to answer is \textit{``How can \acrshort{ais} data combined with specific vessel details be applied to predict future destinations of maritime vessels?''}. To successfully answer the main research question, more sub-questions are to be answered. Moreover, since the thesis aims to apply additional vessel information, mainly vessel segmentation, for the proposed prediction method, the final research question revolves around investigating the possible impact of this information on prediction methods. The full list of research questions are defined as follows:

\begin{enumerate}
    \item How can \acrshort{ais} data combined with specific vessel details be applied to predict future destinations of maritime vessels?
    \begin{enumerate}
    \item What prediction methods can be used to predict vessel destinations?
    \item What information can be used to predict vessel destinations?
    %\item How extensive are existing prediction methods with respect to geographical coverage, specific vessel characteristics, ....?
    \item To what extent do methods proposed in existing work vary in scope of applicability?
    \item How can the validity of predictions made based on different prediction methods be established?
    \end{enumerate}
    \item What is the impact of vessel segmentation by type, size, or capacity on prediction methods, or vessels' general predictability?
    \begin{enumerate}
    \item What types of vessels are more predictable than others?
    \item Do larger vessels travel in more predictable patterns than smaller vessels?
    \end{enumerate}
\end{enumerate}

\section{Planned contributions}

The main contribution of the thesis consists of proposing a generally applicable, global vessel destination prediction method that exceeds existing works' limitations to both geographical and time-related extent. The prediction method also takes advantage of a broader range of specific vessel details in an attempt to achieve higher general prediction accuracy for any type of vessel. The proposed solution includes a method of considering spatial trajectories as well as specific vessel details in a \acrfull{ml} context. Moreover, the developed method provides a foundation that can be flexibly extended by adding more attributes about vessels or voyages to further explore their impact on predictions. To this end, the features used in the proposed solution are further investigated to determine their impact, or importance, and to determine relationships between features and predictability rates.

\section{Remaining thesis structure}

\subsubsection{2. Background}

This chapter aims to give the reader insight into the topic area in a technical sense as well as in the perspective of the shipping industry.

In this chapter, concepts, and terminology relevant to the thesis is explained including technological foundations such as \acrfull{ais}, conceptual foundations such as \acrshort{ais}-based trajectories and trajectory similarity measurements, and techniques applied to predicting future destination ports of traveling vessels, namely, \acrfull{ml}.

\subsubsection{3. Related work}

In this section, related work and literature are presented and discussed in the form of a systematic literature review to establish the extent to which the current state of the art provides insight into the research questions listed in \cref{sec:research_questions}.

\subsubsection{4. Methodology}

In this chapter, the methodology of the proposed solution is explained in detail, as well as the development process and findings discovered when arriving at the proposed solution. This chapter is divided into sequential sections that each describe a step in the process used to compose the \acrshort{ml} training dataset, the formulation of the analytical problem to be solved, and the \acrfull{ml} related data preparation, training, and evaluation.

\subsubsection{5. Results}

In this chapter, the results from the proposed solution are described in detail. It describes the results from the different stages throughout the development process and presents the final results and metrics from the trained \acrfull{ml} model. Furthermore, insights and interpretation of the results are gathered from shipping industry experts to qualitatively assess the validity of the proposed solution.

\subsubsection{6. Discussion}

In this chapter, a summary of the thesis is provided, followed by discussions regarding the proposed solution, the field of study, possible applications, and the approach's validity in terms of both academic and commercial values. Finally, limitations of the thesis and proposed future work are presented and discussed.
